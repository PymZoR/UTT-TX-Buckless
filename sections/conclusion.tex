\paragraph{}
    Aujourd'hui, il existe une grande diversité d'outils permettant de faciliter la production d'une
    documentation claire et complète. Nous avons vu qu'il était possible de faire en sorte d'automatiser
    un maximum l'utilisation de ces outils, afin de les intégrer au coeur du processus de développement
    logiciel.

\paragraph{}
    Le monde du logiciel libre est intimement lié avec la mise en oeuvre de stratégie de documentation
    efficace. En effet, l'attractivité du logiciel envers ses contributeurs est directement corrélé
    avec la présence d'une documentation complète minimisant le temps de formation.
    Au travers de l'exemple de Nylas, il a été force de constater que la documentation était au coeur
    de leur processus de développement. L'approche utilisée, consistant à l'automatisation à l'extrême
    par l'incorporation de la documentation au sein même du code, permet de lier le développeur
    inévitablement a une production de documentation.

\paragraph{}
    La définition d'une stratégie de documentation pour le projet Buckless a été faite en se basant
    sur les différents outils présentés dans l'état de l'art, mais aussi en prenant comme exemple
    des projets open source similaires comme Nylas N1. Il est cependant important de noter que malgré
    la simmilitude des projets au niveau technologique, certaines spécificités on fait que le modèle
    n'a pas pu être tout simplement calqué. Ainsi, il a été choisi d'explorer l'utilisation du
    diagramme de composants en tant qu'outil de représentation de système orienté service.
    L'intérêt de ce diagramme est double, puisqu'il permet de représenter à la fois les services
    web, mais aussi l'interaction des interne de chaque clients avec ces derniers.
    En effet, les technologies web se tournent de plus en plus vers l'utilisation de WebComponents
    réutilisables, appelant des services de manière indépendante. Il devient alors possible de représenter
    deux échelles d'un système sur un même diagramme. On regrette malheureusement l'absence d'outils
    permettant une intégration de la production de diagramme avec celle du code.

\paragraph{}
    A titre plus personnel, ce travail de recherche m'a permis de me rendre compte de l'importance
    de la documentation d'un projet. J'ai avons eu l'occasion d'avoir des discussions techniques avec
    des tiers sur le fonctionnement de Buckless, et la présence d'une documentation forte,
    aussi bien textuelle que schématique a grandement facilité la communication. De plus, la grande
    diversité des outils et leur utilisation pas forcément automatique m'a fait prendre conscience
    de l'actualité du sujet. Il est important de faire de la veille sur les différentes stratégies de
    documentation, car elles sont la porte d'entrée à la production d'un logiciel de bonne qualité.
    Enfin, la formalisation d'une ligne directrice en terme de documentation, et donc de développement,
    a donné un vrai coup de fouet au projet, le rendant plus rigoureux et attractif. Je suis désormais
    confiant quant au temps de formation que devrait suivre un nouveau collaborateur.