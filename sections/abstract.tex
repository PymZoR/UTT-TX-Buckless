Modéliser et documenter un logiciel constituent un travail essentiel pour la maintenance et l'évolution de projet
logiciel sur le long terme. Au delà des document de conception initiaux qui constituent un pré-requis de tout projet
d'ingénierie, un cas extrême de ce besoin de documentation est notamment fourni par le développement de logiciel
open-source qui ajoute la contrainte de rendre disponible le code source et d'ouvrir à la participation d'autres
développeurs au delà de l'équipe projet.

La modélisation UML centrale pour le paradigme objet, les wiki, et framework
de test de logiciel sont des ressources importantes pour développer une documentation claire et exploitable. Cependant,
avec le développement d'application logiciels toujours plus complexes et distribuées désormais permises par
l'Internet, aucune stratégie de documentation claire et spécifique n'émerge pour le développement des applications
orientées services.

L'enjeu de ce document est de proposer et de mettre en oeuvre une stratégie de documentation pour la main-
tenance et l'évolution d'applications distribuées et orientées services. Ce travail sappuiera sur le cas de Buckless,
une application de paiement électronique dématérialisé destinée principalement à la simplification des transactions
dans le cadre des associations étudiantes. Une version majeure de cette solution open source sera bientôt mise en
production par une équipe de développeurs et l'enjeu est de constituer une documentation pertinente pour assurer
la maintenance et l'évolution du projet sur le long terme.

Plusieurs outils sont considérés (modèles UML, outils de documentation d'API, wiki) afin de documenter les
aspects statiques, les principales interfaces et dépendances, et les aspects comportementaux du système. A partir
de l'exemples d'autres cas de projets open-source, une stratégie de documentation sera ainsi formalisée et mise en
oeuvre. Cette dernière est mise à l'épreuve en interrogeant son utilité lors de l'ajout de nouveaux composants et
fonctionnalité au projet Buckless.